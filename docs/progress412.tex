\documentclass{article}
\usepackage[colorlinks=true]{hyperref}
\usepackage{pdfpages}

\begin{document}

\begin{center}
\textbf{Database Systems - Final Project \\
        Progress Report 4/12 \\
        Mark Hudnall and Sam Konowitch
        }
\end{center}

We have been making good progress on our project. We have finished our ER diagram, 
converted it to a relational model, normalized our tables and implemented the 
finalized relational model as a {\tt sql } file, {\tt schema.sql}. Our original
relational model yielded 14 relations. After normalization, our finalized design
now has 16 tables. We are a little concerned that we did not have enough functional
dependencies in our original design, as we performed decomposition on only two tables.
We are brainstorming different fields to add to our relations that would make for
more difficult decompositions and a more robust model of athletic teams and workouts. 

We may schedule a meeting early next week (or send you an email) if we rack our brains
and don't come up with anything non-trivial. We do have a good amount of tables, but 
we're not sure if our schema/normalization is complex enough.

That said, we have implemented the schema in SQL and created our database without errors.
Moving forward, we both have some extra time this weekend. We hope first to do  
some brainstorming to make our schema more robust, so we can show off some more 
complex decompositions. We'll then begin working on our web component (still this weekend).
By April 16, we hope to complete the interface for coaches and general user registration. 
We do not think that the web programming will be particularly difficult, as we both
have some experience with web programming.

Additionally, we said during our meeting that we were deciding between Python and Java for
our server language. We have decided to use Python, with a simple web framework called
Flask. We both have experience with Python and Flask. Flask is also very easy to get up 
and running with.  For our database interfacing, we are using a module called mysql-python, 
which is very similar to JDBC. We will provide detailed instructions on how to install and run
our server environment on a local machine (Python makes it very easy with the {\tt pip} tool). 

Although we have spent some time designing our schema, we think that the web programming
portion of the project will be straightforward (we both used Python and web frameworks
for last weekend's hackathon, so we're feeling pretty comfortable with these tools). 
We would love feedback on our database design, and especially the decompositions, but 
bear in mind that we plan to expand our design and incorporate more FDs and decompositions
to make our application more robust. 

We're also using Github to version control, so you can check out our schema, ER diagram, 
and project proposal (where we have been documenting our efforts) 
\href{https://github.com/landakram/database-final}{here}. This document will likely be 
incorporated into the proposal in the near future.


\end{document}
